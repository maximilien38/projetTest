\documentclass{report}[12pt]

\usepackage[utf8]{inputenc}
\usepackage[T1]{fontenc}
\usepackage[francais]{babel}
\usepackage{url}
\usepackage{color, soul}
\usepackage{multicol}
\usepackage{verbatim}
\usepackage{amsmath, amssymb, amsfonts}
\usepackage{graphicx}
\usepackage[hidelinks]{hyperref}
\usepackage{ulem}
\usepackage{geometry}
\usepackage{longtable}
\usepackage{multirow}
\usepackage{lscape}
\usepackage{tipa}
\usepackage{listings}
\lstset{frame=tb,
 language=Java,
 aboveskip=3mm,
 belowskip=3mm,
 showstringspaces=false,
 columns=flexible,
 basicstyle={\small\ttfamily},
 numbers=none,
 numberstyle=\tiny,
 keywordstyle=\color{blue},
 breaklines=true,
 breakatwhitespace=true,
 tabsize=3
}

\title{JML - TD 4}
\author{Maximilien \bsc{Faure} - Alexandre \bsc{Sahuc}}

\renewcommand{\thesection}{\Roman{section}}
\renewcommand{\thesubsection}{\Roman{section}.\Roman{subsection}}

\begin{document}

\maketitle

\section{Préconditions et invariants}
\subsection{Invariants}
\paragraph{Invariant 1}
\begin{lstlisting}
(0 <= nb_inc && nb_inc < 50);
\end{lstlisting}
Le nombre de règles d'incompatibilité doit être compris entre 0 et 49 pour ne pas dépasser du tableau.

\paragraph{Invariant 2}
\begin{lstlisting}
(0 <= nb_assign && nb_assign < 30);
\end{lstlisting}
Le nombre d'assignement doit être compris entre 0 et 29 pour ne pas dépasser du tableau.

\paragraph{Invariant 3}
\begin{lstlisting}
(\forall int i; 0 <= i && i < nb_inc; 
	incomp[i][0].startsWith("Prod") && incomp[i][1].startsWith("Prod"));
\end{lstlisting}
Une incompatibilité associe deux produits.

\paragraph{Invariant 4}
\begin{lstlisting}
(\forall int i; 0 <= i && i < nb_assign; 
	assign[i][0].startsWith("Bat") && assign[i][1].startsWith("Prod"));
\end{lstlisting}
Chaque assignement doit être entre un bâtiment et un produit (d'abord le bâtiment puis le produit).

\paragraph{Invariant 5}
\begin{lstlisting}
(\forall int i; 0 <= i && i < nb_inc; !(incomp[i][0]).equals(incomp[i][1]));
\end{lstlisting}
Un produit ne peut pas être incompatible avec lui-même.

\paragraph{Invariant 6}
\begin{lstlisting}
(\forall int i; 0 <= i && i < nb_inc; 
	(\exists int j; 0 <= j && j < nb_inc; 
		(incomp[i][0]).equals(incomp[j][1]) 
			&& (incomp[j][0]).equals(incomp[i][1]))); 
\end{lstlisting}
La relation d'incompatibilité est réflexive.

\paragraph{Invariant 7}
\begin{lstlisting}
(\forall int i; 0 <= i &&  i < nb_assign; 
	(\forall int j; 0 <= j && j < nb_assign; 
		(i != j && (assign[i][0]).equals(assign [j][0])) ==>
		(\forall int k; 0 <= k && k < nb_inc;
			(!(assign[i][1]).equals(incomp[k][0])) 
				|| (!(assign[j][1]).equals(incomp[k][1])))));
\end{lstlisting}
Deux produits incompatibles ne peuvent être placés dans un même bâtiment.

\subsection{Préconditions}
\paragraph{add\_incomp}
\begin{lstlisting}
requires nb_inc >= 0; 
requires nb_inc < 49;
\end
Le nombre d'incompatibilités déclarées avant l'appel à la fonction doit être compris entre 0 et 48 (respect de l'invariant 1)
\begin{lstlisting}
requires !(prod1.equals(prod2));
\end{lstlisting}
Les deux produits déclarés comme incompatibles doivent être différents (par respect de l'invariant 5).
\begin{lstlisting}
requires prod1.startsWith("Prod") && prod2.startsWith("Prod")
\end{lstlisting}
Les deux arguments passées en paramètre doivent représenter deux produits (respect de l'invariant 3).
\begin{lstlisting}
requires 
	(\forall int i; 0 <= i && i < nb_assign;
		(\forall int j; 0 <= j && j < nb_assign;
      		(i != j && (assign[i][0]).equals(assign[j][0])) && (!(assign[i][1].equals(assign[j][1]))) ==>
      			(!((assign[i][1].equals(prod1) || assign[i][1].equals(prod2)) &&
      			   (assign[j][1].equals(prod1) || assign[j][1].equals(prod2))))));
\end{lstlisting}
Deux produits déclarés incompatibles ne peuvent pas avoir été placés dans un même bâtiment auparavant (invariant 7).

\paragraph{add\_assign}
\begin{lstlisting}
requires nb_assign < 29;
requires nb_assign >= 0;
\end{lstlisting}
Le nombre d'assignments déclarés avant l'appel à la fonction doit être compris entre 0 et 28 (respect de l'invariant 2).
\begin{lstlisting}
requires bat.startsWith("Bat");
requires prod.startsWith("Prod");
\end{lstlisting}
Les paramètres doivent être un bâtiment et un produit (respect de l'invariant 4).
\begin{lstlisting}
requires
    (\forall int i; 0 <= i &&  i < nb_assign; 
        (bat.equals(assign [i][0])) ==> 
			(\forall int k; 0 <= k && k < nb_inc;
            	(!prod.equals(incomp[k][0])) || !(assign[i][1]).equals(incomp[k][1])));
  
\end{lstlisting}
Le produit \verb-prod- placé dans le bâtiment \verb-bat- ne peut pas être incompatible avec d'autres produits de ce bâtiment (respect de l'invariant 7).

\paragraph{compatible}
\begin{lstlisting}
requires prod1.stratsWith("Prod");
requires prod2.startsWith("Prod");
\end{lstlisting}
Les deux arguments doivents être des produits.

\begin{lstlisting}
ensures \result == true ==> 
	(\forall int i; 0 <= i && i < nb_inc;
		((!incomp[i][0].equals(prod1))  || (!incomp[i][1].equals(prod2))));
\end{lstlisting}
La fonction renvoie \verb-true- si les deux produits ne sont pas incompatibles.

\paragraph{findBat}
Pour la fonction \verb-findBat-, nous avons décidé de renvoyer le bâtiment où est stocké le produit si ce dernier a déjà été stocké.
Sinon, nous renvoyons le premier bâtiment rencontré où le produit peut être stocké.
Si, dans tous les bâtiments où un produit à déjà été assigné, il n'y a aucun bâtiment qui peut contenir le produit passé en paramètre, la fonction renvoie \verb-null-.
\begin{lstlisting}
ensures \result != null <==>
			(\forall int i; 0 <= i && i < nb_assign;
				(!assign[i][0].equals(\result)) || compatible(assign[i][1], prod));
\end{lstlisting}
Si le résultat est différent de null (donc un bâtiement a été trouvé), le produit n'est pas incompatible avec les autres produits de ce bâtiment.
\end{document}
